% Options for packages loaded elsewhere
\PassOptionsToPackage{unicode}{hyperref}
\PassOptionsToPackage{hyphens}{url}
%
\documentclass[
]{article}
\usepackage{lmodern}
\usepackage{amssymb,amsmath}
\usepackage{ifxetex,ifluatex}
\ifnum 0\ifxetex 1\fi\ifluatex 1\fi=0 % if pdftex
  \usepackage[T1]{fontenc}
  \usepackage[utf8]{inputenc}
  \usepackage{textcomp} % provide euro and other symbols
\else % if luatex or xetex
  \usepackage{unicode-math}
  \defaultfontfeatures{Scale=MatchLowercase}
  \defaultfontfeatures[\rmfamily]{Ligatures=TeX,Scale=1}
\fi
% Use upquote if available, for straight quotes in verbatim environments
\IfFileExists{upquote.sty}{\usepackage{upquote}}{}
\IfFileExists{microtype.sty}{% use microtype if available
  \usepackage[]{microtype}
  \UseMicrotypeSet[protrusion]{basicmath} % disable protrusion for tt fonts
}{}
\makeatletter
\@ifundefined{KOMAClassName}{% if non-KOMA class
  \IfFileExists{parskip.sty}{%
    \usepackage{parskip}
  }{% else
    \setlength{\parindent}{0pt}
    \setlength{\parskip}{6pt plus 2pt minus 1pt}}
}{% if KOMA class
  \KOMAoptions{parskip=half}}
\makeatother
\usepackage{xcolor}
\IfFileExists{xurl.sty}{\usepackage{xurl}}{} % add URL line breaks if available
\IfFileExists{bookmark.sty}{\usepackage{bookmark}}{\usepackage{hyperref}}
\hypersetup{
  pdftitle={MLMC Code Demo},
  pdfauthor={Howard Thom},
  hidelinks,
  pdfcreator={LaTeX via pandoc}}
\urlstyle{same} % disable monospaced font for URLs
\usepackage[margin=1in]{geometry}
\usepackage{color}
\usepackage{fancyvrb}
\newcommand{\VerbBar}{|}
\newcommand{\VERB}{\Verb[commandchars=\\\{\}]}
\DefineVerbatimEnvironment{Highlighting}{Verbatim}{commandchars=\\\{\}}
% Add ',fontsize=\small' for more characters per line
\usepackage{framed}
\definecolor{shadecolor}{RGB}{248,248,248}
\newenvironment{Shaded}{\begin{snugshade}}{\end{snugshade}}
\newcommand{\AlertTok}[1]{\textcolor[rgb]{0.94,0.16,0.16}{#1}}
\newcommand{\AnnotationTok}[1]{\textcolor[rgb]{0.56,0.35,0.01}{\textbf{\textit{#1}}}}
\newcommand{\AttributeTok}[1]{\textcolor[rgb]{0.77,0.63,0.00}{#1}}
\newcommand{\BaseNTok}[1]{\textcolor[rgb]{0.00,0.00,0.81}{#1}}
\newcommand{\BuiltInTok}[1]{#1}
\newcommand{\CharTok}[1]{\textcolor[rgb]{0.31,0.60,0.02}{#1}}
\newcommand{\CommentTok}[1]{\textcolor[rgb]{0.56,0.35,0.01}{\textit{#1}}}
\newcommand{\CommentVarTok}[1]{\textcolor[rgb]{0.56,0.35,0.01}{\textbf{\textit{#1}}}}
\newcommand{\ConstantTok}[1]{\textcolor[rgb]{0.00,0.00,0.00}{#1}}
\newcommand{\ControlFlowTok}[1]{\textcolor[rgb]{0.13,0.29,0.53}{\textbf{#1}}}
\newcommand{\DataTypeTok}[1]{\textcolor[rgb]{0.13,0.29,0.53}{#1}}
\newcommand{\DecValTok}[1]{\textcolor[rgb]{0.00,0.00,0.81}{#1}}
\newcommand{\DocumentationTok}[1]{\textcolor[rgb]{0.56,0.35,0.01}{\textbf{\textit{#1}}}}
\newcommand{\ErrorTok}[1]{\textcolor[rgb]{0.64,0.00,0.00}{\textbf{#1}}}
\newcommand{\ExtensionTok}[1]{#1}
\newcommand{\FloatTok}[1]{\textcolor[rgb]{0.00,0.00,0.81}{#1}}
\newcommand{\FunctionTok}[1]{\textcolor[rgb]{0.00,0.00,0.00}{#1}}
\newcommand{\ImportTok}[1]{#1}
\newcommand{\InformationTok}[1]{\textcolor[rgb]{0.56,0.35,0.01}{\textbf{\textit{#1}}}}
\newcommand{\KeywordTok}[1]{\textcolor[rgb]{0.13,0.29,0.53}{\textbf{#1}}}
\newcommand{\NormalTok}[1]{#1}
\newcommand{\OperatorTok}[1]{\textcolor[rgb]{0.81,0.36,0.00}{\textbf{#1}}}
\newcommand{\OtherTok}[1]{\textcolor[rgb]{0.56,0.35,0.01}{#1}}
\newcommand{\PreprocessorTok}[1]{\textcolor[rgb]{0.56,0.35,0.01}{\textit{#1}}}
\newcommand{\RegionMarkerTok}[1]{#1}
\newcommand{\SpecialCharTok}[1]{\textcolor[rgb]{0.00,0.00,0.00}{#1}}
\newcommand{\SpecialStringTok}[1]{\textcolor[rgb]{0.31,0.60,0.02}{#1}}
\newcommand{\StringTok}[1]{\textcolor[rgb]{0.31,0.60,0.02}{#1}}
\newcommand{\VariableTok}[1]{\textcolor[rgb]{0.00,0.00,0.00}{#1}}
\newcommand{\VerbatimStringTok}[1]{\textcolor[rgb]{0.31,0.60,0.02}{#1}}
\newcommand{\WarningTok}[1]{\textcolor[rgb]{0.56,0.35,0.01}{\textbf{\textit{#1}}}}
\usepackage{graphicx,grffile}
\makeatletter
\def\maxwidth{\ifdim\Gin@nat@width>\linewidth\linewidth\else\Gin@nat@width\fi}
\def\maxheight{\ifdim\Gin@nat@height>\textheight\textheight\else\Gin@nat@height\fi}
\makeatother
% Scale images if necessary, so that they will not overflow the page
% margins by default, and it is still possible to overwrite the defaults
% using explicit options in \includegraphics[width, height, ...]{}
\setkeys{Gin}{width=\maxwidth,height=\maxheight,keepaspectratio}
% Set default figure placement to htbp
\makeatletter
\def\fps@figure{htbp}
\makeatother
\setlength{\emergencystretch}{3em} % prevent overfull lines
\providecommand{\tightlist}{%
  \setlength{\itemsep}{0pt}\setlength{\parskip}{0pt}}
\setcounter{secnumdepth}{-\maxdimen} % remove section numbering

\title{MLMC Code Demo}
\author{Howard Thom}
\date{04/02/2021}

\begin{document}
\maketitle

\hypertarget{introduction}{%
\subsection{Introduction}\label{introduction}}

Script to demonstrate EVPPI using MLMC for a simple net benefit function
This can be applied to any economic model implemented in R. The primary
modifications are to the generate\_input\_parameters and
generate\_net\_benefit functions. The first creates a data frame of
sampled parameters; the function must include the option to keep all or
a subset of sampled parameters fixed for the whole sample. The latter
function converts these sampled parameters into net benefits.

Following these primary modifications, the remaining steps are simple.
Modify the EVPPI\_x\_std\_p function, which uses the two primary
functions to generate net benefits for a number of samples M*N, and
EVPPI\_x\_l\_p, which acts as a wrapper for the general level l
estimating function EVPPI\_l\_p and passing it EVPPI\_x\_std\_p.

Finally, run mlmc.test with EVPPI\_x\_l\_p as an input and other
specifications for the MLMC algorithm.

The code is explained below.

The example economic mdoel has a net benefit function for two decision
options \[{NB_{1}(x,y) = 400x+200y}\] \[{NB_{0}(x,y) = 0}\] And these
call two parameters \[x\sim Normal(0, 2),  y\sim Normal(2, 4)\] \#\# R
Code

First load the necessary packages. Note that bcea is loaded only for
comparison with the MLMC estimates.

\begin{Shaded}
\begin{Highlighting}[]
\KeywordTok{require}\NormalTok{(grid)}
\KeywordTok{require}\NormalTok{(Rcpp)}
\KeywordTok{require}\NormalTok{(doRNG)}
\KeywordTok{library}\NormalTok{(BCEA)}
\KeywordTok{require}\NormalTok{(mlmc)}
\end{Highlighting}
\end{Shaded}

\hypertarget{mlmc-source-files}{%
\section{MLMC source files}\label{mlmc-source-files}}

These are based on GPL-2 `Matlab' code by Mike Giles
(\url{http://people.maths.ox.ac.uk/~gilesm/mlmc/}) Authors are Louis
Aslett
\href{mailto:aslett@stats.ox.ac.uk}{\nolinkurl{aslett@stats.ox.ac.uk}},
Mike Giles
\href{mailto:Mike.Giles@maths.ox.ac.uk}{\nolinkurl{Mike.Giles@maths.ox.ac.uk}},
and Tigran Nagapetyan
\href{mailto:nagapetyan@stats.ox.ac.uk}{\nolinkurl{nagapetyan@stats.ox.ac.uk}}

\begin{Shaded}
\begin{Highlighting}[]
\CommentTok{# MLMC code to calculate the value to some accuracy}
\KeywordTok{source}\NormalTok{(}\StringTok{"mlmc.R"}\NormalTok{)}
\CommentTok{# Test function of the convergence rates and output the MLMC results}
\KeywordTok{source}\NormalTok{(}\StringTok{"mlmc.test.R"}\NormalTok{) }
\end{Highlighting}
\end{Shaded}

\hypertarget{model-definition}{%
\section{Model definition}\label{model-definition}}

Here we define global parameters and the economic model functions. This
is where the primary modifications are needed to run this script on
other models.

\begin{Shaded}
\begin{Highlighting}[]
\CommentTok{# Global options}
\NormalTok{n_samples =}\StringTok{ }\DecValTok{1000} \CommentTok{# Only used for EVPI}
\NormalTok{n_treatment <-}\StringTok{ }\DecValTok{2}

\CommentTok{# Net benefit functions are  NB1(x,y) = 400x+200y, NB0(x,y) = 0}
\NormalTok{generate_net_benefit <-}\StringTok{ }\ControlFlowTok{function}\NormalTok{(n_samples, input_parameters) \{}
\NormalTok{  NetB <-}\StringTok{ }\KeywordTok{matrix}\NormalTok{(}\DataTypeTok{nrow =}\NormalTok{ n_samples, }\DataTypeTok{ncol =}\NormalTok{ n_treatment) }
\NormalTok{  NetB[, }\DecValTok{1}\NormalTok{] <-}\StringTok{ }\DecValTok{400} \OperatorTok{*}\StringTok{ }\NormalTok{input_parameters}\OperatorTok{$}\NormalTok{x }\OperatorTok{+}\StringTok{ }\DecValTok{200} \OperatorTok{*}\StringTok{ }\NormalTok{input_parameters}\OperatorTok{$}\NormalTok{y}
\NormalTok{  NetB[, }\DecValTok{2}\NormalTok{] <-}\StringTok{ }\KeywordTok{rep}\NormalTok{(}\DecValTok{0}\NormalTok{, n_samples)}
  
  \KeywordTok{return}\NormalTok{(NetB)}
\NormalTok{\}}

\CommentTok{# Two random parameters x~Normal(0, 2), y~Normal(2, 4)}
\NormalTok{generate_input_parameters <-}\StringTok{ }\ControlFlowTok{function}\NormalTok{(n_samples, }\DataTypeTok{hold_constant =} \KeywordTok{c}\NormalTok{()) \{}
\NormalTok{  input_parameters <-}\StringTok{ }\KeywordTok{as.data.frame}\NormalTok{(}\KeywordTok{matrix}\NormalTok{(}\DataTypeTok{nrow =}\NormalTok{ n_samples, }\DataTypeTok{ncol =} \DecValTok{2}\NormalTok{))}
  \KeywordTok{colnames}\NormalTok{(input_parameters) <-}\StringTok{ }\KeywordTok{c}\NormalTok{(}\StringTok{"x"}\NormalTok{, }\StringTok{"y"}\NormalTok{)}
  
  \CommentTok{# Only generate one random value for parameters in the hold_constant vector}
  \ControlFlowTok{if}\NormalTok{(}\OperatorTok{!}\KeywordTok{is.element}\NormalTok{(}\StringTok{"x"}\NormalTok{, hold_constant)) \{}
\NormalTok{    input_parameters[, }\StringTok{"x"}\NormalTok{] <-}\StringTok{ }\KeywordTok{rnorm}\NormalTok{(n_samples, }\DataTypeTok{mean =} \DecValTok{0}\NormalTok{, }\DataTypeTok{sd =} \KeywordTok{sqrt}\NormalTok{(}\DecValTok{2}\NormalTok{))}
\NormalTok{  \} }\ControlFlowTok{else}\NormalTok{ \{}
\NormalTok{    input_parameters[, }\StringTok{"x"}\NormalTok{] <-}\StringTok{ }\KeywordTok{rnorm}\NormalTok{(}\DecValTok{1}\NormalTok{, }\DataTypeTok{mean =} \DecValTok{0}\NormalTok{, }\DataTypeTok{sd =} \KeywordTok{sqrt}\NormalTok{(}\DecValTok{2}\NormalTok{))}
\NormalTok{  \}}
  \ControlFlowTok{if}\NormalTok{(}\OperatorTok{!}\KeywordTok{is.element}\NormalTok{(}\StringTok{"y"}\NormalTok{, hold_constant)) \{}
\NormalTok{    input_parameters[, }\StringTok{"y"}\NormalTok{] <-}\StringTok{ }\KeywordTok{rnorm}\NormalTok{(n_samples, }\DataTypeTok{mean =} \DecValTok{2}\NormalTok{, }\DataTypeTok{sd =} \KeywordTok{sqrt}\NormalTok{(}\DecValTok{4}\NormalTok{))}
\NormalTok{  \} }\ControlFlowTok{else}\NormalTok{ \{}
\NormalTok{    input_parameters[, }\StringTok{"y"}\NormalTok{] <-}\StringTok{ }\KeywordTok{rnorm}\NormalTok{(}\DecValTok{1}\NormalTok{, }\DataTypeTok{mean =} \DecValTok{2}\NormalTok{, }\DataTypeTok{sd =} \KeywordTok{sqrt}\NormalTok{(}\DecValTok{4}\NormalTok{))}
\NormalTok{  \}}
  
  \KeywordTok{return}\NormalTok{(input_parameters)}
\NormalTok{\}}
\end{Highlighting}
\end{Shaded}

\hypertarget{wrappers-for-parameter-and-net-benefit-functions}{%
\section{Wrappers for parameter and net benefit
functions}\label{wrappers-for-parameter-and-net-benefit-functions}}

The two wrappers below are needed for EVPPI of \(x\) and EVPPI of \(y\)

\begin{Shaded}
\begin{Highlighting}[]
\CommentTok{# Wrapper function to generate the net benefit holding the parameters of interest constant}
\CommentTok{# Repeats parameters of interest M times and generates random for remainder}
\CommentTok{# Calculates net benefit based on these}
\NormalTok{EVPPI_x_std_p<-}\ControlFlowTok{function}\NormalTok{(M,N)}
\NormalTok{\{}
  \CommentTok{# N is the number of outer samples, M=2^l is the number of inner samples}
  \CommentTok{# Total number of samples is NN}
\NormalTok{  NN <-}\StringTok{ }\NormalTok{M}\OperatorTok{*}\NormalTok{N}
  
\NormalTok{  input_parameters <-}\StringTok{ }\KeywordTok{generate_input_parameters}\NormalTok{(}\DataTypeTok{n_samples =}\NormalTok{ NN, }\DataTypeTok{hold_constant =} \KeywordTok{c}\NormalTok{(}\StringTok{"x"}\NormalTok{))}
\NormalTok{  NetB <-}\StringTok{ }\KeywordTok{generate_net_benefit}\NormalTok{(}\DataTypeTok{n_samples =}\NormalTok{ NN, }\DataTypeTok{input_parameters =}\NormalTok{ input_parameters)}
  
  \KeywordTok{return}\NormalTok{(NetB)}
\NormalTok{\}}

\CommentTok{# Correpsonding wrapper function for y}
\NormalTok{EVPPI_y_std_p<-}\ControlFlowTok{function}\NormalTok{(M,N)}
\NormalTok{\{}
\NormalTok{  NN <-}\StringTok{ }\NormalTok{M}\OperatorTok{*}\NormalTok{N}
\NormalTok{  input_parameters <-}\StringTok{ }\KeywordTok{generate_input_parameters}\NormalTok{(}\DataTypeTok{n_samples =}\NormalTok{ NN, }\DataTypeTok{hold_constant =} \KeywordTok{c}\NormalTok{(}\StringTok{"y"}\NormalTok{))}
\NormalTok{  NetB <-}\StringTok{ }\KeywordTok{generate_net_benefit}\NormalTok{(}\DataTypeTok{n_samples =}\NormalTok{ NN, }\DataTypeTok{input_parameters =}\NormalTok{ input_parameters)}
  \KeywordTok{return}\NormalTok{(NetB)}
\NormalTok{\}}
\end{Highlighting}
\end{Shaded}

\hypertarget{mlmc-level-l-estimator}{%
\section{MLMC level l estimator}\label{mlmc-level-l-estimator}}

Function to provide level \(l\) difference estimate using estimator
\(d_\ell^{(n)}\) based on parameters and net benefit function.

See the main paper for details but the fine \(l\) estimator is \[
e_{\ell}^{(n)} = \max_{d\in D}\frac{1}{2^\ell}\sum_{m=1}^{2^\ell}f_d(X^{(n)},Y^{(n,m)}) - \max_{d\in D}\frac{1}{2^\ell N}\sum_{i=1}^{N}\sum_{m=1}^{2^\ell}f_d(X^{(i)},Y^{(i,m)}).
\] While the coarse \(l-1\) estimator is \[
e_{\ell-1}^{(n)}\ = \  \frac{1}{2}\bigg[\max_{d\in D}\frac{1}{2^{\ell-1}} \sum_{m=1}^{2^{\ell-1}} f_d(X^{(n)},Y^{(n,m)})
+ \max_{d\in D}\frac{1}{2^{\ell-1}} \sum_{m=2^{\ell-1}+1}^{2^\ell}f_d(X^{(n)},Y^{(n,m)}) \bigg]\\
 - \max_{d\in D}\frac{1}{2^\ell N}\sum_{i=1}^{N}\sum_{m=1}^{2^\ell}f_d(X^{(i)},Y^{(i,m)})
\]

The difference estimator \(d_\ell^{(n)}\) is then \[
d_\ell^{(n)}\ = \ \max_{d\in D}\frac{1}{2^\ell}\sum_{m=1}^{2^\ell}f_d(X^{(n)},Y^{(n,m)})\nonumber\\
  - \frac{1}{2}\bigg[\max_{d\in D}\frac{1}{2^{\ell-1}}\sum_{m=1}^{2^{\ell-1}} f_d(X^{(n)},Y^{(n,m)})
+ \max_{d\in D}\frac{1}{2^{\ell-1}}\sum_{m=2^{\ell-1}+1}^{2^\ell}f_d(X^{(n)},Y^{(n,m)}) \bigg]
\]

The corresponding MLMC estimator of DIFF (=EVPI-EVPPI) is \[
\widehat{\mathrm{DIFF}}_\ell^*(N_1,N_2,...,N_L) = - \sum_{\ell=1}^L d_\ell^{(n)}
\]

\begin{Shaded}
\begin{Highlighting}[]
\NormalTok{EVPPI_l_p<-}\ControlFlowTok{function}\NormalTok{(l,N, }\DataTypeTok{EVPPI_std_p =} \OtherTok{NULL}\NormalTok{)}
\NormalTok{\{}
  \ControlFlowTok{if}\NormalTok{(}\KeywordTok{is.null}\NormalTok{(EVPPI_std_p)) }\KeywordTok{return}\NormalTok{(}\StringTok{"EVPPI_std_p must be supplied"}\NormalTok{)}
  \CommentTok{# This function generates all the random samples for the EVPPI calculation}
  \CommentTok{# Then calculates the net benefit function}
  
  \CommentTok{# N is the number of outer samples, M=2^l is the number of inner samples}
\NormalTok{  sum1 <-}\StringTok{ }\KeywordTok{rep}\NormalTok{(}\DecValTok{0}\NormalTok{, }\DecValTok{7}\NormalTok{)}
  \CommentTok{# Need to store results of different stats}
\NormalTok{  M =}\StringTok{ }\DecValTok{2}\OperatorTok{^}\NormalTok{(l }\OperatorTok{+}\StringTok{ }\DecValTok{1}\NormalTok{) }
\NormalTok{  Np =}\StringTok{ }\KeywordTok{max}\NormalTok{(M, }\DecValTok{128}\NormalTok{)}
  
\NormalTok{  inputs =}\StringTok{ }\DecValTok{1}\OperatorTok{:}\KeywordTok{ceiling}\NormalTok{(M }\OperatorTok{*}\StringTok{ }\NormalTok{N }\OperatorTok{/}\StringTok{ }\NormalTok{Np)}
  
  \CommentTok{# Iterate over the inputs}
  \CommentTok{# Can add parallel computation here}
\NormalTok{  results <-}\StringTok{ }\KeywordTok{foreach}\NormalTok{(}\DataTypeTok{i=}\NormalTok{inputs,}\DataTypeTok{.export=}\KeywordTok{c}\NormalTok{(}\StringTok{"n_samples"}\NormalTok{, }\StringTok{"n_treatment"}\NormalTok{,}
                                        \StringTok{"generate_net_benefit"}\NormalTok{, }\StringTok{"generate_input_parameters"}\NormalTok{,}
                                        \StringTok{'EVPPI_x_std_p'}\NormalTok{),}\DataTypeTok{.packages=}\StringTok{'MASS'}\NormalTok{) }\OperatorTok\StringTok{ }\NormalTok{\{}
\NormalTok{                                          NN=}\KeywordTok{min}\NormalTok{(Np, N}\OperatorTok{*}\NormalTok{M}\OperatorTok{-}\NormalTok{(i}\DecValTok{-1}\NormalTok{)}\OperatorTok{*}\NormalTok{Np)}
                                          \CommentTok{#####################################################}
                                          \KeywordTok{EVPPI_std_p}\NormalTok{(M,NN}\OperatorTok{/}\NormalTok{M)}
                                          
                                          \CommentTok{###########################################33########}
\NormalTok{                                        \}   }
  \CommentTok{#NetB <- EVPPI_x_std_p(M, N)}

\CommentTok{# Convert results of foreach to NetB matrix}
  \CommentTok{# If not interested in parallelisation could merge the foreach and for loop to simplify}
\NormalTok{  NetB =}\StringTok{ }\KeywordTok{matrix}\NormalTok{(}\OtherTok{NA}\NormalTok{, M}\OperatorTok{*}\NormalTok{N, n_treatment)}
  \ControlFlowTok{for}\NormalTok{(i }\ControlFlowTok{in}\NormalTok{ inputs)\{}
\NormalTok{    NN <-}\StringTok{ }\KeywordTok{min}\NormalTok{(Np, N}\OperatorTok{*}\NormalTok{M}\OperatorTok{-}\NormalTok{(i}\DecValTok{-1}\NormalTok{)}\OperatorTok{*}\NormalTok{Np)}
\NormalTok{    nn =}\StringTok{ }\KeywordTok{min}\NormalTok{(i}\OperatorTok{*}\NormalTok{Np,N}\OperatorTok{*}\NormalTok{M)}
\NormalTok{    NetB[(nn}\OperatorTok{-}\NormalTok{NN}\OperatorTok{+}\DecValTok{1}\NormalTok{)}\OperatorTok{:}\NormalTok{nn,] =}\StringTok{ }\KeywordTok{matrix}\NormalTok{(}\KeywordTok{unlist}\NormalTok{(results[i]),NN,n_treatment)}
\NormalTok{  \}}
  \CommentTok{# Net benefit based on partial perfect information for every sample}
\NormalTok{  NetB_max =}\StringTok{ }\KeywordTok{apply}\NormalTok{(NetB,}\DecValTok{1}\NormalTok{,max)}
  \CommentTok{# Expected value of max over each set of inner samples}
\NormalTok{  NetB_max_sample =}\StringTok{ }\KeywordTok{apply}\NormalTok{(}\KeywordTok{matrix}\NormalTok{(NetB_max,N,M,}\DataTypeTok{byrow=}\OtherTok{TRUE}\NormalTok{),}\DecValTok{1}\NormalTok{,mean)}
  \CommentTok{# Matrices needed for antithetic variable variance reduction}
\NormalTok{  NetB_low_c_}\DecValTok{1}\NormalTok{ =}\StringTok{ }\KeywordTok{matrix}\NormalTok{(}\OtherTok{NA}\NormalTok{,N,n_treatment)}
\NormalTok{  NetB_low_c_}\DecValTok{2}\NormalTok{ =}\StringTok{ }\KeywordTok{matrix}\NormalTok{(}\OtherTok{NA}\NormalTok{,N,n_treatment)}
\NormalTok{  NetB_low_f =}\StringTok{ }\KeywordTok{matrix}\NormalTok{(}\OtherTok{NA}\NormalTok{,N,n_treatment)}
  \ControlFlowTok{for}\NormalTok{(n }\ControlFlowTok{in} \DecValTok{1}\OperatorTok{:}\NormalTok{n_treatment)\{}
    \CommentTok{# Net benefts for treatment n in matrix of size outer by inner samples}
\NormalTok{    temp =}\StringTok{ }\KeywordTok{matrix}\NormalTok{(NetB[,n],N,M,}\DataTypeTok{byrow=}\OtherTok{TRUE}\NormalTok{)}
    \CommentTok{# Average net benefit over each set of inner samples}
\NormalTok{    NetB_low_f[,n] =}\StringTok{ }\KeywordTok{apply}\NormalTok{(temp,}\DecValTok{1}\NormalTok{,mean)}
    \CommentTok{# Antithetic variable construction splits samples into first and second halves}
    \CommentTok{# Split formula into cases l=0 and l>0}
    \ControlFlowTok{if}\NormalTok{(M}\OperatorTok{==}\DecValTok{2}\NormalTok{)\{}
\NormalTok{      NetB_low_c_}\DecValTok{1}\NormalTok{[,n] =}\StringTok{ }\NormalTok{temp[,}\DecValTok{1}\NormalTok{]}
\NormalTok{      NetB_low_c_}\DecValTok{2}\NormalTok{[,n] =}\StringTok{ }\NormalTok{temp[,}\DecValTok{2}\NormalTok{]}
\NormalTok{    \}}\ControlFlowTok{else}\NormalTok{\{}
      \ControlFlowTok{if}\NormalTok{(N}\OperatorTok{==}\DecValTok{1}\NormalTok{)\{}
\NormalTok{        NetB_low_c_}\DecValTok{1}\NormalTok{[,n] =}\StringTok{ }\KeywordTok{sum}\NormalTok{(temp[,}\DecValTok{1}\OperatorTok{:}\KeywordTok{max}\NormalTok{(M}\OperatorTok{/}\DecValTok{2}\NormalTok{,}\DecValTok{2}\NormalTok{)])}\OperatorTok{/}\KeywordTok{max}\NormalTok{(M}\OperatorTok{/}\DecValTok{2}\NormalTok{,}\DecValTok{2}\NormalTok{)}
\NormalTok{        NetB_low_c_}\DecValTok{2}\NormalTok{[,n] =}\StringTok{ }\KeywordTok{sum}\NormalTok{(temp[,}\KeywordTok{min}\NormalTok{(M}\OperatorTok{/}\DecValTok{2}\OperatorTok{+}\DecValTok{1}\NormalTok{,M}\DecValTok{-1}\NormalTok{)}\OperatorTok{:}\NormalTok{M])}\OperatorTok{/}\NormalTok{(M}\OperatorTok{-}\KeywordTok{min}\NormalTok{(M}\OperatorTok{/}\DecValTok{2}\OperatorTok{+}\DecValTok{1}\NormalTok{,M}\DecValTok{-1}\NormalTok{)}\OperatorTok{+}\DecValTok{1}\NormalTok{)  }
\NormalTok{      \}}\ControlFlowTok{else}\NormalTok{\{}
\NormalTok{        NetB_low_c_}\DecValTok{1}\NormalTok{[,n] =}\StringTok{ }\KeywordTok{rowSums}\NormalTok{(temp[,}\DecValTok{1}\OperatorTok{:}\KeywordTok{max}\NormalTok{(M}\OperatorTok{/}\DecValTok{2}\NormalTok{,}\DecValTok{2}\NormalTok{)])}\OperatorTok{/}\KeywordTok{max}\NormalTok{(M}\OperatorTok{/}\DecValTok{2}\NormalTok{,}\DecValTok{2}\NormalTok{)}
\NormalTok{        NetB_low_c_}\DecValTok{2}\NormalTok{[,n] =}\StringTok{ }\KeywordTok{rowSums}\NormalTok{(temp[,}\KeywordTok{min}\NormalTok{(M}\OperatorTok{/}\DecValTok{2}\OperatorTok{+}\DecValTok{1}\NormalTok{,M}\DecValTok{-1}\NormalTok{)}\OperatorTok{:}\NormalTok{M])}\OperatorTok{/}\NormalTok{(M}\OperatorTok{-}\KeywordTok{min}\NormalTok{(M}\OperatorTok{/}\DecValTok{2}\OperatorTok{+}\DecValTok{1}\NormalTok{,M}\DecValTok{-1}\NormalTok{)}\OperatorTok{+}\DecValTok{1}\NormalTok{)        }
\NormalTok{      \}}

\NormalTok{    \}}
\NormalTok{  \}}
\NormalTok{  NetB_low_f_sample =}\StringTok{ }\KeywordTok{apply}\NormalTok{(NetB_low_f,}\DecValTok{1}\NormalTok{,max)}
  \CommentTok{# Put antithetic variable construction together}
\NormalTok{  NetB_low_c_sample =}\StringTok{ }\NormalTok{(}\KeywordTok{apply}\NormalTok{(NetB_low_c_}\DecValTok{1}\NormalTok{,}\DecValTok{1}\NormalTok{,max)}\OperatorTok{+}\KeywordTok{apply}\NormalTok{(NetB_low_c_}\DecValTok{2}\NormalTok{,}\DecValTok{1}\NormalTok{,max))}\OperatorTok{/}\DecValTok{2}
  \CommentTok{# Fine estimator (i.e. e_l^(n))}
\NormalTok{  Pf =}\StringTok{ }\NormalTok{NetB_max_sample }\OperatorTok{-}\StringTok{ }\NormalTok{NetB_low_f_sample}
  \CommentTok{# Coarse estimator (i.e. e_(l-1)^(n))}
\NormalTok{  Pc =}\StringTok{ }\NormalTok{NetB_max_sample }\OperatorTok{-}\StringTok{ }\NormalTok{NetB_low_c_sample}
  
  \CommentTok{# Sum the moments of the estimator }
  \CommentTok{# First is the mean of the difference estimator d_l^(n)}
  \CommentTok{# Summing these difference estimates gives an estimate of DIFF= EVPI-EVPPI}
\NormalTok{  sum1[}\DecValTok{1}\NormalTok{] =}\StringTok{ }\NormalTok{sum1[}\DecValTok{1}\NormalTok{] }\OperatorTok{+}\StringTok{ }\KeywordTok{sum}\NormalTok{(Pf}\OperatorTok{-}\NormalTok{Pc);}
\NormalTok{  sum1[}\DecValTok{2}\NormalTok{] =}\StringTok{ }\NormalTok{sum1[}\DecValTok{2}\NormalTok{] }\OperatorTok{+}\StringTok{ }\KeywordTok{sum}\NormalTok{((Pf}\OperatorTok{-}\NormalTok{Pc)}\OperatorTok{^}\DecValTok{2}\NormalTok{);}
\NormalTok{  sum1[}\DecValTok{3}\NormalTok{] =}\StringTok{ }\NormalTok{sum1[}\DecValTok{3}\NormalTok{] }\OperatorTok{+}\StringTok{ }\KeywordTok{sum}\NormalTok{((Pf}\OperatorTok{-}\NormalTok{Pc)}\OperatorTok{^}\DecValTok{3}\NormalTok{);}
\NormalTok{  sum1[}\DecValTok{4}\NormalTok{] =}\StringTok{ }\NormalTok{sum1[}\DecValTok{4}\NormalTok{] }\OperatorTok{+}\StringTok{ }\KeywordTok{sum}\NormalTok{((Pf}\OperatorTok{-}\NormalTok{Pc)}\OperatorTok{^}\DecValTok{4}\NormalTok{);}
\NormalTok{  sum1[}\DecValTok{5}\NormalTok{] =}\StringTok{ }\NormalTok{sum1[}\DecValTok{5}\NormalTok{] }\OperatorTok{+}\StringTok{ }\KeywordTok{sum}\NormalTok{(Pf);}
\NormalTok{  sum1[}\DecValTok{6}\NormalTok{] =}\StringTok{ }\NormalTok{sum1[}\DecValTok{6}\NormalTok{] }\OperatorTok{+}\StringTok{ }\KeywordTok{sum}\NormalTok{(Pf}\OperatorTok{^}\DecValTok{2}\NormalTok{);}
\NormalTok{  sum1[}\DecValTok{7}\NormalTok{] =}\StringTok{ }\NormalTok{sum1[}\DecValTok{7}\NormalTok{] }\OperatorTok{+}\StringTok{ }\NormalTok{M}\OperatorTok{*}\NormalTok{N;}
  
  \KeywordTok{return}\NormalTok{(sum1)}
\NormalTok{\}}
\end{Highlighting}
\end{Shaded}

\hypertarget{wrapper-for-evppi_l_p}{%
\section{Wrapper for EVPPI\_l\_p}\label{wrapper-for-evppi_l_p}}

These functions are needed for the EVPPI of \(x\) and EVPPI of \(y\) and
pass the necessary EVPPI\_std\_p to the level l estimator EVPPI\_l\_p

\begin{Shaded}
\begin{Highlighting}[]
\CommentTok{# Wrapper function for EVPPI of x}
\NormalTok{EVPPI_x_l_p <-}\StringTok{ }\ControlFlowTok{function}\NormalTok{(}\DataTypeTok{l =}\NormalTok{ l,}\DataTypeTok{N =}\NormalTok{ N) \{}
  \KeywordTok{return}\NormalTok{(}\KeywordTok{EVPPI_l_p}\NormalTok{(l, N, }\DataTypeTok{EVPPI_std_p =}\NormalTok{ EVPPI_x_std_p))}
\NormalTok{\}}
\CommentTok{# Wrapper function for EVPPI of y}
\NormalTok{EVPPI_y_l_p <-}\StringTok{ }\ControlFlowTok{function}\NormalTok{(}\DataTypeTok{l =}\NormalTok{ l,}\DataTypeTok{N =}\NormalTok{ N) \{}
  \KeywordTok{return}\NormalTok{(}\KeywordTok{EVPPI_l_p}\NormalTok{(l, N, }\DataTypeTok{EVPPI_std_p =}\NormalTok{ EVPPI_y_std_p))}
\NormalTok{\}}
\end{Highlighting}
\end{Shaded}

\hypertarget{evpi-and-evppi-by-regression}{%
\section{EVPI and EVPPI by
regression}\label{evpi-and-evppi-by-regression}}

For comparison, we use GP model from BCEA to estimate the EVPPI of \(x\)
and \(y\)

\begin{Shaded}
\begin{Highlighting}[]
\CommentTok{# EVPI is about 121.08}
\NormalTok{bcea_object}\OperatorTok{$}\NormalTok{evi}
\end{Highlighting}
\end{Shaded}

\begin{verbatim}
## [1] 124.7058
\end{verbatim}

\begin{Shaded}
\begin{Highlighting}[]
\CommentTok{# EVPPI of x is about 63/74/91/75/65  Average 73.6}
\NormalTok{bcea_evppi_x}\OperatorTok{$}\NormalTok{evppi}
\end{Highlighting}
\end{Shaded}

\begin{verbatim}
## [1] 84.82218
\end{verbatim}

\begin{Shaded}
\begin{Highlighting}[]
\CommentTok{# EVPPI for y is about 23/34/46/37/20 Average 32}
\NormalTok{bcea_evppi_y}\OperatorTok{$}\NormalTok{evppi}
\end{Highlighting}
\end{Shaded}

\begin{verbatim}
## [1] 35.57351
\end{verbatim}

\hypertarget{evppi-of-x-using-mlmc}{%
\section{EVPPI of x using MLMC}\label{evppi-of-x-using-mlmc}}

Now use the mlmc.test() function to estimate the EVPPI of \(x\).

M refinement cost factor \(2^{\gamma}\) in the general MLMC Throrem) N
number of samples to use in the tests L number of levels to use in the
tests N0 initial number of samples which are used for the first 3 levels
and for any subsequent levels which are automatically added. Must be
\textgreater0 eps.v a vector of all the target accuracies in the tests.
Must all be \textgreater0 Lmin the minimum level of refinement. Must be
\(\ge 2\) Lmax the maximum level of refinement. Must be \(\ge\) Lmin

\begin{Shaded}
\begin{Highlighting}[]
\KeywordTok{set.seed}\NormalTok{(}\DecValTok{33}\NormalTok{)}
\NormalTok{tst_x <-}\StringTok{ }\KeywordTok{mlmc.test}\NormalTok{(EVPPI_x_l_p, }\DataTypeTok{M=}\DecValTok{2}\NormalTok{, }\DataTypeTok{N=}\DecValTok{1024}\NormalTok{,}
                          \DataTypeTok{L=}\DecValTok{4}\NormalTok{, }\DataTypeTok{N0=}\DecValTok{1024}\NormalTok{,}
                          \DataTypeTok{eps.v=}\KeywordTok{c}\NormalTok{(}\DecValTok{60}\NormalTok{,}\DecValTok{30}\NormalTok{,}\DecValTok{15}\NormalTok{,}\DecValTok{7}\NormalTok{,}\DecValTok{3}\NormalTok{,}\DecValTok{1}\NormalTok{),}
                          \DataTypeTok{Lmin=}\DecValTok{2}\NormalTok{, }\DataTypeTok{Lmax=}\DecValTok{10}\NormalTok{)}
\end{Highlighting}
\end{Shaded}

\begin{verbatim}
## 
## **********************************************************
## *** Convergence tests, kurtosis, telescoping sum check ***
## **********************************************************
## 
##  l   ave(Pf-Pc)    ave(Pf)   var(Pf-Pc)    var(Pf)    kurtosis     check 
## ---------------------------------------------------------------------------
\end{verbatim}

\begin{verbatim}
## Warning: executing %dopar% sequentially: no parallel backend registered
\end{verbatim}

\begin{verbatim}
##  0   2.3338e+01  2.3338e+01  3.3343e+03  3.3343e+03  0.0000e+00  0.0000e+00 
##  1   1.0059e+01  2.9487e+01  9.4862e+02  2.5981e+03  1.8955e+01  2.9895e-01 
##  2   5.0736e+00  3.8240e+01  3.4724e+02  2.3560e+03  2.6048e+01  3.3219e-01 
##  3   2.6050e+00  4.0921e+01  1.3008e+02  2.2648e+03  4.1764e+01  7.4623e-03 
##  4   1.2841e+00  4.4510e+01  4.4248e+01  2.2174e+03  5.7800e+01  2.4267e-01 
## 
## ******************************************************
## *** Linear regression estimates of MLMC parameters ***
## ******************************************************
## 
##  alpha in 0.991101  (exponent for (MLMC weak convergence)
##  beta  in 1.486120  (exponent for (MLMC variance) 
##  gamma in 1.000000  (exponent for (MLMC cost) 
## 
## ***************************** 
## *** MLMC complexity tests *** 
## ***************************** 
## 
##   eps       value   mlmc_cost   std_cost  savings     N_l 
## --------------------------------------------------------- 
## 60.0000  3.4719e+01  1.434e+04  6.981e+00     0.00      1024      1024      1024
## 30.0000  3.8124e+01  1.434e+04  2.792e+01     0.00      1024      1024      1024
## 15.0000  3.5871e+01  1.434e+04  1.117e+02     0.01      1024      1024      1024
## 7.0000  3.0580e+01  1.461e+04  9.860e+02     0.07      1024      1024      1024        17
## 3.0000  4.2076e+01  1.919e+04  1.051e+04     0.55      1677      1024      1024       136        43
## 1.0000  4.0766e+01  1.424e+05  3.784e+05     2.66     15466      7219      2711      1424       548       196        63
\end{verbatim}

\begin{Shaded}
\begin{Highlighting}[]
\CommentTok{# EVPPI is lowest eps estimate }
\CommentTok{# Remember MLMC estimates DIFF from EVPI so need to subtract this from total EVPPI}
\CommentTok{# The EVPI in this case using 10^7 samples is 121}
\DecValTok{121} \OperatorTok{-}\StringTok{ }\NormalTok{tst_x}\OperatorTok{$}\NormalTok{P[}\KeywordTok{length}\NormalTok{(tst_x}\OperatorTok{$}\NormalTok{P)]}
\end{Highlighting}
\end{Shaded}

\begin{verbatim}
## [1] 80.23447
\end{verbatim}

\hypertarget{evppi-of-y-using-mlmc}{%
\section{EVPPI of y using MLMC}\label{evppi-of-y-using-mlmc}}

Now use the mlmc.test() function to estimate the EVPPI of \(y\).

\begin{Shaded}
\begin{Highlighting}[]
\CommentTok{# Calculate the EVPPI for y}
\KeywordTok{set.seed}\NormalTok{(}\DecValTok{123}\NormalTok{)}
\NormalTok{tst_y <-}\StringTok{ }\KeywordTok{mlmc.test}\NormalTok{(EVPPI_y_l_p, }\DataTypeTok{M=}\DecValTok{2}\NormalTok{, }\DataTypeTok{N=}\DecValTok{1024}\NormalTok{,}
                   \DataTypeTok{L=}\DecValTok{4}\NormalTok{, }\DataTypeTok{N0=}\DecValTok{1024}\NormalTok{,}
                   \DataTypeTok{eps.v=}\KeywordTok{c}\NormalTok{(}\DecValTok{60}\NormalTok{,}\DecValTok{30}\NormalTok{,}\DecValTok{15}\NormalTok{,}\DecValTok{7}\NormalTok{),}
                  \DataTypeTok{Lmin=}\DecValTok{2}\NormalTok{, }\DataTypeTok{Lmax=}\DecValTok{10}\NormalTok{)}
\end{Highlighting}
\end{Shaded}

\begin{verbatim}
## 
## **********************************************************
## *** Convergence tests, kurtosis, telescoping sum check ***
## **********************************************************
## 
##  l   ave(Pf-Pc)    ave(Pf)   var(Pf-Pc)    var(Pf)    kurtosis     check 
## ---------------------------------------------------------------------------
##  0   4.7667e+01  4.7667e+01  8.7237e+03  8.7237e+03  0.0000e+00  0.0000e+00 
##  1   2.5744e+01  7.2458e+01  3.3345e+03  7.1803e+03  1.2056e+01  4.3084e-02 
##  2   1.0585e+01  7.5732e+01  9.8151e+02  5.3056e+03  2.1614e+01  4.1284e-01 
##  3   6.4307e+00  8.5748e+01  4.6089e+02  4.9403e+03  2.4915e+01  2.3240e-01 
##  4   3.2644e+00  8.8795e+01  1.7613e+02  4.3195e+03  3.9201e+01  1.5534e-02 
## 
## ******************************************************
## *** Linear regression estimates of MLMC parameters ***
## ******************************************************
## 
##  alpha in 0.848566  (exponent for (MLMC weak convergence)
##  beta  in 1.239168  (exponent for (MLMC variance) 
##  gamma in 1.000000  (exponent for (MLMC cost) 
## 
## ***************************** 
## *** MLMC complexity tests *** 
## ***************************** 
## 
##   eps       value   mlmc_cost   std_cost  savings     N_l 
## --------------------------------------------------------- 
## 60.0000  8.1438e+01  1.434e+04  1.572e+01     0.00      1024      1024      1024
## 30.0000  8.1093e+01  1.434e+04  6.288e+01     0.00      1024      1024      1024
## 15.0000  9.3721e+01  1.539e+04  1.638e+03     0.11      1024      1024      1024        24        11         5
## 7.0000  9.1478e+01  1.626e+04  3.761e+03     0.23      1024      1024      1024        66        27
\end{verbatim}

\begin{Shaded}
\begin{Highlighting}[]
\CommentTok{# EVPPI is lowest eps estimate }
\CommentTok{# Remember MLMC estimates DIFF from EVPI so need to subtract this from total EVPPI}
\CommentTok{# The EVPI in this case using 10^7 samples is 121}
\DecValTok{121} \OperatorTok{-}\StringTok{ }\NormalTok{tst_y}\OperatorTok{$}\NormalTok{P[}\KeywordTok{length}\NormalTok{(tst_y}\OperatorTok{$}\NormalTok{P)]}
\end{Highlighting}
\end{Shaded}

\begin{verbatim}
## [1] 29.52207
\end{verbatim}

\end{document}
